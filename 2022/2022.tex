\documentclass[10pt,ngerman]{scrartcl}
\usepackage{lmodern}
\renewcommand{\sfdefault}{lmss}
\renewcommand{\ttdefault}{lmtt}
\usepackage[T1]{fontenc}
\usepackage[utf8]{inputenc}
\usepackage[a4paper]{geometry}
\geometry{verbose,tmargin=2.8cm,bmargin=4cm,lmargin=2.3cm,rmargin=2.3cm,headheight=0cm,headsep=1cm,columnsep=1cm}
\setcounter{secnumdepth}{-2}
\usepackage{babel}
\usepackage{array}
\usepackage{textcomp}
\usepackage{enumitem}
\usepackage{setspace}
\usepackage{tabularx}
\usepackage{multicol}
\usepackage[dvipsnames,svgnames,x11names,hyperref]{xcolor}
\usepackage[unicode=true,pdfusetitle,
 bookmarks=true,bookmarksnumbered=false,bookmarksopen=false,
 breaklinks=false,pdfborder={0 0 0},pdfborderstyle={},backref=false,colorlinks=true]
 {hyperref}

\makeatletter

%%%%%%%%%%%%%%%%%%%%%%%%%%%%%% LyX specific LaTeX commands.
%% Because html converters don't know tabularnewline
\providecommand{\tabularnewline}{\\}

%%%%%%%%%%%%%%%%%%%%%%%%%%%%%% Textclass specific LaTeX commands.
\newlength{\lyxlabelwidth}      % auxiliary length 

%%%%%%%%%%%%%%%%%%%%%%%%%%%%%% User specified LaTeX commands.
\setlist[itemize,1]
{itemsep=0.2em,
parsep=0pt,
partopsep=0pt,
topsep=0.3em}
\setlist[itemize,2]
{itemsep=0.1em,
parsep=0pt,
partopsep=0pt,
topsep=0.1em}
\usepackage{scrlayer-scrpage}
\pagestyle{scrheadings}
\chead{Tutoreninformation Studier Langsam}

\RedeclareSectionCommand[
    beforeskip=.7\baselineskip,
    afterskip=.1\baselineskip]{subsection}
\addtokomafont{subsection}{\large}
\setlength{\parindent}{0pt}

\RedeclareSectionCommand[
    afterskip=.6\baselineskip]{section}

\AtBeginDocument{
    \def\labelitemi{\(\triangleright\)}
}

\makeatother

\newcommand{\multilinecell}[2]{\begin{tabular}{@{}#1@{}}#2\end{tabular}}

\begin{document}

%%%%%%%%%%%%%%%%%%%%%%%%%%%%%%%%%%%%%%%%%%%%%%%%%%%%%%%%%%%%%%%%%%%%%%%%% START

\title{\vspace{-1em}
    Tutoreninfo Studier Langsam 2022\vspace{-0.8em}
}
\maketitle

\section{Räume und Leute}\label{rooms}

% \renewcommand\tabularxcolumn[1]{m{#1}}
\begin{tabularx}{\columnwidth}[H]{r|XXX}
    & \textbf{Euler} & \textbf{Lovelace} & \textbf{Turing} \\
    \hline
    Leute
    & \textbf{Jonas S.}, \textbf{Katharina}, Louis, Tamira, Sarah, Leo, Jens, Cián, Johanna, Yannick, Paul L., Piotr
    & \textbf{Alina}, \textbf{Linus}, Laura, Max, Lea, Deborah, Jonas L., Jonathan, Florian B., Anh, Simon
    & \textbf{Dominik}, \textbf{Nadine}, Linda, Karina, Falko, Luca, Peter, Paul Z., Jan-Arne, Leon, Florian G. \\
    \hline
    Mo (sW, 14-20) & \multicolumn{3}{c}{alle zusammen -1.012, -1.013} \\
    Mi (ganztags) & \multicolumn{1}{c}{-1.009} & \multicolumn{1}{c}{-1.011} & \multicolumn{1}{c}{-1.012} \\
    Do (20-) & \multicolumn{3}{c}{alle zusammen -1.011, -1.012, -1.013, -1.014} \\
\end{tabularx}
\vspace{1em}


\begin{multicols}{2}
\section{Allgemeine Informationen}

% TODO: Regeln der FS, wie funzt das mit den Getränken, etc.

\subsection{Telefonnummern}

\begin{tabular}[H]{ll}
    Hauptorga & \href{tel:+49 721 48074683}{+49 721 48074683}
    % HERE ist Linus grad noch am machen
\end{tabular}



\subsection{Getränke}\label{drinks}

Linus hat Getränke betellt und die Bezahlung vorgestreckt.
Zum Abholen im Mathebau einfach in der Fachschaft fragen, eine weitere Zahlung ist nicht nötig.
Insbesondere am Montag könnte es sinnvoll sein, vorher anzurufen und sich anzukündigen, damit das schneller geht.

Wie viel für wann geplant ist, ist \href{https://docs.google.com/spreadsheets/d/1p1hiGPHs2fquxRcn8Yz74vaicJAu-LVF3NOmNHuNc6g/edit?usp=sharing}{dieser Tabelle} zu entnehmen.
Die Bestellung ist auf Provision, sodass wir nicht Benötigtes vollständig erstattet bekommen.
Sollten wir merken, dass es knapp wird, können wir auch noch ``Nachbestellen'' bzw. Reserven der Fachschaft nutzen, die dann natürlich auch gezahlt werden müssen.
Bitte achtet darauf, dass die Pfandflaschen zurückgehen, da wir sonst kein Pfand mehr zurückbekommen.
Weißt bitte auch die Erstis darauf hin.

\section{Wochenplan}

\subsection{Übersicht}

Auf der \href{https://studierlangsam.de/wochenplan}{Website} oder auch auf der \hyperref[LastPage]{letzen Seite}.


% Nachfolgend für jeden Tag: Tabelle mit Terminen (Wann, Was, Wo), danach Details

\subsection{Montag}
% Dominik

\subsubsection{Begrüßungsveranstaltung}

% TODO (alt)
Fabian, Pablo und Jonas S. kaufen Kekse, Getränke (Apfelsaftschorle,
Sprudel), Servietten, Kreppband, Eddings und Pappteller und -becher
für das Pizzaessen. Laura nimmt der Fachschaft die Seminarräume ab.
Alle vier treffen sich am Mathebau und tragen den Einkäufe in die
Räume. Sie bringen Kekse und Schild an den Audimax.

Die Begrüßungsveranstaltung findet um 9:00 Uhr im Audimax statt. Wir
dürfen U-Boote sein (müssen aber nicht!), sollten dann aber in der
Rolle des Erstis bleiben. Falls es nicht genug Platz für alle Erstis
gibt, sollten die U-Boote natürlich wieder gehen (Codewort dafür ist
,,InWis raus``). Um 9:30 Uhr treffen wir uns an dem uns zugewiesenen
Platz hinter dem Audimax. Um 10:00 Uhr beginnt die Vorstellung der
Gruppen (Svenja, Moritz und Joshua stellen vor). Wir sind als siebtes
dran, nach Lila Pause. Anschließend gehen alle (nur die angemeldeten!)
Tutoren gemeinsam auf die Bühne.

Nach der Veranstaltung warten Valentin, Nadine (Moritz), Titia und
Svenja (offiziell nur eine Person) mit Keksen und Schild am Nordeingang
des Audimax (am uns von der Fachschaft zugewiesenen Ort), um Erstis
aufzugabeln und zum Rest der Gruppe zu bringen. Johannes und Johanna
finden einen Platz für die Kennenlernspiele und teilen diesen per
Slack mit. Wenn alle da sind, laufen wir geschlossen zu den gefundenen
Plätze. Falko zählt die Erstis. Laura und Nadine ordern mögichst
früh entsprechend viel Pizza für ca. 13:30 Uhr (ggf. später).

Erstis Während des Wartens mögichst begrüßen und \emph{in Gespräche einbinden}.

\subsubsection{Kennenlernen}

Wir erklären, dass es zwei Gruppen gibt und teilen die Erstis enstprechend
auf. Es darf noch beliebig die Gruppe gewechselt werden. Die Gruppen
teilen sich auf die verschiedenen Plätze auf. Die Erstis schreiben
sich Namensschilder. Wir spielen wir die üblichen Kennenlernspiele:
\begin{itemize}
    \item In Ecken stellen
    \item Sortiert aufstellen
    \item Bingo (finde Menschen, mit bestimmter Eigenschaft)
    \item Nach Bedarf:
          \begin{itemize}
              \item Atomspiel
              \item gordischer Knoten
          \end{itemize}
\end{itemize}
Johanna und Johannes leiten die Spiele.

Kurz bevor die Pizza kommt suchen wir uns einen Platz zum Pizzaessen.
Es folgen erste Infos:
\begin{itemize}
    \item Programm, Kontakt
    \item Fahrräder mitbringen!
    \item Mittwoch Morgen Teller, Tasse und Besteck mitbringen.
    \item Cocktailabend Freitag
    \item Es gibt Erstishirts (Verkauf bei Grillen durch Fachschaft)
    \item Kleidung für Olympia
    \item Bild hochladen für KIT-Card
\end{itemize}
Es folgt das Pizzaessen. Luca, Michael, Johannes und Wendy nehmen
nebenbei die Daten (Name, Email, Cocktailabend, Pizza, Einverständnis
die Daten in Airtable zu speichern) der Erstis in Airtable auf und
sammeln das Geld ein. Die Zeit bis zur Campus-Tour wird mit Gesprächen
und wenn nötig Activity überbrückt. Preise für die Erstis:

\medskip{}

\begin{tabular}{ll}
    Pizza \& Sonstiges             & 5€ + 4€\tabularnewline
    Cocktailabend mit/ohne Alkohol & 10€ / 7€\tabularnewline
    Erstishirt bei Fachschaft      & 5€\tabularnewline
\end{tabular}


\subsubsection{Campus-Tour}

% TODO - Alina
% inkl aktuellen Hörsälen

Der Anfang der Tour ist für jede Gruppe gleich:
\begin{description}
    \item [{15:00}] Einteilung in Gruppen\setlength{\itemsep}{0pt}
\end{description}
Jede Gruppe sollte das folgende Programm, aber \emph{in unterschiedlicher Reihenfolge} absolvieren:
\begin{itemize}
    \item Infobau
        \begin{itemize}
            \item Abgabekästen
            \item Info-Fachschaft
            \item Info Bib mit GruppenRäumen
            \item Atis- Accounts machen
            \item Seminarräume
        \end{itemize}
    \item Hörsaal am Fasanengarten
        \begin{itemize}
            \item HM 1 für Infos und LA 1 für Infos(freitags)
        \end{itemize}
    \item Audimax
        \begin{itemize}
            \item GBI und Programmieren
        \end{itemize}
    \item Mensa
        \begin{itemize}
            \item Karte kodieren
            \item Linien-System vorstellen
            \item Auf Cafeteria hinweisen
            \item Anekdotisch: Aufgrund von Stromausfall gab es im letzten Semester ein paar mal Essen umsonst
        \end{itemize}
    \item Bibliothek
        \begin{itemize}
            \item LernRäume
            \item Führungen in der ersten Semesterwoche (dauer ca. 30 min Montags bis Freitags 10-14 Uhr)
        \end{itemize}
    \item AKK
        \begin{itemize}
            \item Kaffee
            \item Günstiges Bier und Glühwein
            \item Bar schichten machen/mithelfen
            \item Partys/Schlonze
            \item Alternativ: AKK Führung (noch nicht angefragt)
            \item o-Phest findet hier statt
        \end{itemize}
    \item Daimler und Benz HS
        \begin{itemize}
            \item Ana 1 und La 1 Mathe(Daimeler) und La 1 mittwochs für infos(Benz)
            \item es ist egal was man hört nur richtige Klausur schreiben
        \end{itemize}
    \item Hertz
        \begin{itemize}
            \item IAM für die Mathematiker die nicht Programmieren hören
            \item Uni Kino
        \end{itemize}
    \item Studeinbüro
        \begin{itemize}
            \item falls noch was zu klären ist geht man am Dienstag mit ein paat Tutoren
                noch hin
        \end{itemize}
    \item Mathebau
        \begin{itemize}
            \item Abgabekästen
            \item FSM
            \item Seminarräume und Lernecken
            \item Räume für studneten sind nach innen (falls ihr nach außen geht für
                ein Tutorium z.b. seid ihr falsch)
            \item Anekdotisch im 3. OG sind die Germanisten
        \end{itemize}
    \item SCC
        \begin{itemize}
            \item Hinweis: Führung am Freitag um 14:30 Uhr
        \end{itemize}
    \item Gaede Hörsal
        \begin{itemize}
            \item IAM (für Mathe die nicht proggen machen
        \end{itemize}
    \item Gerthsen Bei Bedarf
    \item Neue Chemie
        \begin{itemize}
            \item Ana 1
        \end{itemize}
\end{itemize}

\subsubsection{Abendessen}

% TODO (alt)
Wir nehmen nicht am Grillen der Fachschaft teil. für das Abendessen
ist ab 19:00 im Oxford Pub für Turing von Moritz und im Pfannestiel
für Lovelace von Joshua reserviert. Am Ende sollten wir uns sagen
lassen, wie viel Geld wir insgesamt ausgegeben haben, dies für kommende
Reservierungen abschätzen können.

David geht um 18:00 zum Treffen der Fachschaft (HS -101).



\subsection{Dienstag}

\subsubsection{Frühstück}

10 Uhr. \\
Euler: intro CAFÉ \\
Lovelace: cafe palaver \\
Turing: Extrablatt

\subsubsection{Formalitäten}

ca. 13 Uhr. \\
KIT-Karten werden hier abgeholt: \\
Verwaltungsgebäude 10.11, Raum 214 \\
Englerstraße 13 KIT-Karte
(\href{https://goo.gl/maps/qhAjKJah3wpiFEzq5}{Google Maps})

Das ist eine gute Gelegenheit Erstis noch bei anderen liegen gebliebenen Formalitäten zu unterstützen, falls es welche gibt.

\subsubsection{FBI}

Alle Fachbereichsinformationen beginnen um 14 Uhr in diesen Räumen:

\begin{tabularx}{\columnwidth}[H]{ll}
    \hline
    Bachelor Mathematik & Neue Chemie   \\
    Bachelor Informatik & Audimax       \\
    Master Mathematik   & Infobau, -102 \\
    Master Informatik   & Infobau, -101 \\
    \hline
\end{tabularx}

Für Lehramt gibt es gesondert am Mittwoch Programm.

Während des Frühstücks werden für alle FBIs Menschen gefunden, die uns benachrichtigen, wenn sie vorbei sind.
Die Kleingruppen bringen iher Erstis getrennt zu den FBIs.
Treffpunkt für alle, die direkt zur FBI kommen, ist vor dem Audimax.
Diese Erstis könnten auch noch einmal mit einem Schild oder Banner empfangen werden.

\subsubsection{Stadtführung}

% TODO - Cián


\subsection{Mittwoch}

Für heute sind \hyperref[drinks]{Getränke} bei der Fachschaft bestellt und können abgeholt werden.

\subsubsection{Gemeinsames Frühstück}

9 Uhr in den \hyperref[rooms]{Räumen} im Mathebau.

Belag bringen die Erstis mit.

% TODO, wird gerade geklärt (https://discord.com/channels/739522765677133894/1030855474951770154)
% alt
160 Brötchen werden von Laura organisiert. Jonas, Falko, Johannes
und Joshua bringen Waffeleisen mit. Moritz, Svenja, Luca und Joshua
bringen Waffelteig mit. Belag bringen die Erstis mit. Peter und Philip
bringen einen Wasserkocher und Titia und Johanna eine French-Press-Kanne
mit.

\subsubsection{O-Rallye}

11 - 17 Uhr mit den \hyperref[rooms]{Räumen} im Mathebau als Basis.

Vor Start: \\
Auf Kneipentour am Abend hinweisen.
Es wird angekündigt, dass Donnerstag Abend ein Spieleabend stattfinden wird und Erstis Spiele mitbringen können.
Auf Atis-Accounts hinweisen.
Lehramt-Ertis werden erneut auf die gesonderten heutigen \hyperref[lehramt]{Veranstaltungen} und die Anmeldung dafür hingewiesen.

Die Fragebögen für die O-Rallye werden von XXX in der Fachschaft
abgeholt und um 11 Uhr an die Erstis (zur Selbstorganisation) übergeben.
Es sollten immer Tutoren für Rückfragen o.ä. im Seminarraum sein.

\subsubsection{Lehramtinformation}\label{lehramt}

\href{https://www.hoc.kit.edu/zlb/Veranstaltungskalender.php/event/46988?}{Veranstaltungsseite}.
Programm und Anmeldung sind von dort zu erreichen.

11:30 - 16:30 Uhr ist dort Programm, danach Grillen.

Gegebenenfalls werden Erstis von einem Tutor dorthin (Geb. 11.10, Engelbert-Arnold-Hörsaal, \href{https://goo.gl/maps/R9WbmtbrKRxdziYY9}{Google Maps}) geführt.


\subsubsection{Kneipentour}

% Tabelle https://docs.google.com/spreadsheets/d/1Ea5M858ijKzbtYuySIMmbNUEUUtv-4jGpAmmq59vIvc/edit#gid=0
% Hierfür soll eventuell auch noch ein getrenntes Übersichts-Dokument entstehen


Es werden Bars und Kneipen besucht.
Euler startet am Euro, Lovelace am Marktplatz und Turing am Durlacher Tor.
Zur Orientierung dient \href{https://docs.google.com/spreadsheets/d/1Ea5M858ijKzbtYuySIMmbNUEUUtv-4jGpAmmq59vIvc/edit?usp=sharing}{diese Tabelle}.
Wir starten vermutlich in 3-6 Gruppen und werden je nach Abfall immer weiter zusammenführen.
Das Wechseln und der weitere Verlauf werden engmaschig per Chat koordiniert.
Wir versuchen im Shotz zu enden (hat bis 2 Uhr auf).

Als Alternativprogramm finden Spiele im Z10 statt.



\subsection{Donnerstag}

\subsubsection{Frühstück}

10 Uhr. \\
Euler: Extrablatt \\
Lovelace: intro CAFÉ \\
Turing: cafe palaver

\subsubsection{O-Lympia}

% TODO (alt), noch gibt es nicht ausreichend Infos von der Fachschaft
Um 13:30 meldet Moritz die Anzahl unserer Erstis bei der O-Lympia-Orga
an. Alle anderen kommen um kurz vor 13:30 zum Forum. Als Ausweichprogramm
bieten wir Klettern im DAV, Kino, Schwimmen und Spiele im Z10 an.

\subsubsection{Spieleabend}

% TODO wer?
Wir treffen uns um 20 Uhr in den \hyperref[rooms]{Seminarräumen} zum Spieleabend.
XXX besorgen Knabberzeug.
Tutoren und Erstis bringen Spiele mit.



\subsection{Freitag}

\subsubsection{Abschlussveranstaltung}

11:30 - 12:30 Uhr.

Pause für uns.

\subsubsection{Mensa}

12:45 Uhr.

Wir gehen gemeinsam mit unseren Erstis in der Mensa essen.
Einige müssen vermutlich noch ihre Karte kodieren und aufladen.
Daran denken, dazukommende vor der Mensa aufzugabeln.

\subsubsection{Aktivitäten}

Wir bieten einige Aktivitäten an und treffen uns dafür um 14:30 Uhr vor der Mensa.
Folgende Personen kümmern sich um die Durchführung:

\begin{tabular}[H]{ll}
    SCC & Linus \\
    Naturkundemuseum & Laura \\
    ZKM & Jonathan \\
    Bouldern & Jonas, Katha, Piotr, Paul \\
    Minigolf & Anh, Jan-Arne \\
\end{tabular}

Parallel läuft der Aufbau vom Cocktailabend.

\subsubsection{Cocktailabend}

Teilnahme nur nach Anmeldung bis Donnerstag.
Es wird zeitlich passend der \href{https://discord.com/channels/739522765677133894/963505261388107846/1030438323086438440}{Anfahrtsplan} verschickt.

Falko, Flo und Alina gehen ab 13 Uhr mit dem Stadtmobil von der Fachschaft in die Metro und bringen auch die Vorräte mit.
% TODO stimmt so?
Luca hat die K1-Bar gemietet.
Der Aufbau beginnt um 15 Uhr und wird von Max und Luca organisiert.
% TODO Max: wann?
Alle Helfer der Barschichten, sollen ab XX Uhr der Bareinführung lauschen.
\href{https://docs.google.com/spreadsheets/d/17ycbRMmSfck2oAsiZ9djPUULAgf4-vCJ6Q4EXgZmB0g/edit?usp=sharing}{Schichtenplan}.



\subsection{Samstag}

\subsubsection{Mädels-Brunch}

Alle Mädels gerne anwesend.

\subsubsection{Mathe-Treff}

Alle Mathes gerne anwesend.

% Wandern?

\end{multicols}
\label{LastPage}
\end{document}

% TODO zum rum schicken hängen wir einfach den Wochenplan noch hinten dran
