\documentclass[10pt,twocolumn,ngerman]{scrartcl}
\usepackage{lmodern}
\renewcommand{\sfdefault}{lmss}
\renewcommand{\ttdefault}{lmtt}
\usepackage[T1]{fontenc}
\usepackage[utf8]{inputenc}
\usepackage[a4paper]{geometry}
\geometry{verbose,tmargin=2.8cm,bmargin=4cm,lmargin=2.3cm,rmargin=2.3cm,headheight=0cm,headsep=1cm,columnsep=1cm}
\setcounter{secnumdepth}{-2}
\usepackage{babel}
\usepackage{array}
\usepackage{textcomp}
\usepackage{enumitem}
\usepackage{setspace}
\usepackage[unicode=true,pdfusetitle,
 bookmarks=true,bookmarksnumbered=false,bookmarksopen=false,
 breaklinks=false,pdfborder={0 0 0},pdfborderstyle={},backref=false,colorlinks=false]
 {hyperref}

\makeatletter

%%%%%%%%%%%%%%%%%%%%%%%%%%%%%% LyX specific LaTeX commands.
%% Because html converters don't know tabularnewline
\providecommand{\tabularnewline}{\\}

%%%%%%%%%%%%%%%%%%%%%%%%%%%%%% Textclass specific LaTeX commands.
\newlength{\lyxlabelwidth}      % auxiliary length 

%%%%%%%%%%%%%%%%%%%%%%%%%%%%%% User specified LaTeX commands.
\setlist[itemize,1]
{itemsep=0.2em,
parsep=0pt,
partopsep=0pt,
topsep=0.3em}
\setlist[itemize,2]
{itemsep=0.1em,
parsep=0pt,
partopsep=0pt,
topsep=0.1em}
\usepackage{scrlayer-scrpage}
\pagestyle{scrheadings}
\chead{Tutoreninformation Studier Langsam}

\RedeclareSectionCommand[
    beforeskip=.7\baselineskip,
    afterskip=.1\baselineskip]{subsection}
\addtokomafont{subsection}{\large}
\setlength{\parindent}{0pt}

\RedeclareSectionCommand[
    afterskip=.6\baselineskip]{section}

\AtBeginDocument{
    \def\labelitemi{\(\triangleright\)}
}

\makeatother

\begin{document}

%%%%%%%%%%%%%%%%%%%%%%%%%%%%%%%%%%%%%%%%%%%%%%%%%%%%%%%%%%%%%%%%%%%%%%%%% START

\title{\vspace{-1em}
    Tutoreninfo Studier Langsam 2022\vspace{-0.8em}
}
\maketitle

\section{Räume und Leute}

% TODO: Tabelle mit Gruppenaufteilung

\section{Allgemeine Informationen}

% TODO: Regeln der FS, wie funzt das mit den Getränken, etc.

\section{Wochenplan}

\subsection{Übersicht}

% TODO: Tabelle, die wir auch an die FS schicken


% Nachfolgend für jeden Tag: Tabelle mit Terminen (Wann, Was, Wo), danach Details

\subsection{Montag}

% TODO Tabelle
% Anm. d. L.: Was im Namen des automatischen Konverters ist das hier denn für ein Unfall?!
\begin{spacing}{0.8}
    \textsf{\footnotesize{}}%
    \begin{tabular*}{1\columnwidth}{@{\extracolsep{\fill}}>{\raggedright}p{0.1\columnwidth}>{\raggedright}p{0.45\columnwidth}>{\raggedright}p{0.35\columnwidth}}
        \textsf{\footnotesize{}9:00} & \textsf{\footnotesize{}Begrüßungsveranstaltung} & \textsf{\footnotesize{}Audimax}\tabularnewline[0.3em]
        \textsf{\footnotesize{}9:30} & \textsf{\footnotesize{}Treffen der Tutoren} & \textsf{\footnotesize{}neben dem Audimax}\tabularnewline[0.3em]
        \textsf{\footnotesize{}10:00} & \textsf{\footnotesize{}Tutorenvorstellung} & \textsf{\footnotesize{}Audimax}\tabularnewline[0.3em]
        \textsf{\footnotesize{}\dots{}} & \textsf{\footnotesize{}Kennenlernen, Pizza essen} & \textsf{\footnotesize{}Innenhof \& Seminarräume}\tabularnewline[0.3em]
        \textsf{\footnotesize{}15:00} & \textsf{\footnotesize{}Campus-Tour} & \tabularnewline[0.3em]
        \textsf{\footnotesize{}19:00} & \textsf{\footnotesize{}Abendessen} & \textsf{\footnotesize{}Oxford Pub (Turing), Pfannestiel (Lovelace)}\tabularnewline[0.3em]
    \end{tabular*}{\footnotesize\par}
\end{spacing}

\subsubsection{Begrüßungsveranstaltung}

% TODO (alt)
Fabian, Pablo und Jonas S. kaufen Kekse, Getränke (Apfelsaftschorle,
Sprudel), Servietten, Kreppband, Eddings und Pappteller und -becher
für das Pizzaessen. Laura nimmt der Fachschaft die Seminarräume ab.
Alle vier treffen sich am Mathebau und tragen den Einkäufe in die
Räume. Sie bringen Kekse und Schild an den Audimax.

Die Begrüßungsveranstaltung findet um 9:00 Uhr im Audimax statt. Wir
dürfen U-Boote sein (müssen aber nicht!), sollten dann aber in der
Rolle des Erstis bleiben. Falls es nicht genug Platz für alle Erstis
gibt, sollten die U-Boote natürlich wieder gehen (Codewort dafür ist
,,InWis raus``). Um 9:30 Uhr treffen wir uns an dem uns zugewiesenen
Platz hinter dem Audimax. Um 10:00 Uhr beginnt die Vorstellung der
Gruppen (Svenja, Moritz und Joshua stellen vor). Wir sind als siebtes
dran, nach Lila Pause. Anschließend gehen alle (nur die angemeldeten!)
Tutoren gemeinsam auf die Bühne.

Nach der Veranstaltung warten Valentin, Nadine (Moritz), Titia und
Svenja (offiziell nur eine Person) mit Keksen und Schild am Nordeingang
des Audimax (am uns von der Fachschaft zugewiesenen Ort), um Erstis
aufzugabeln und zum Rest der Gruppe zu bringen. Johannes und Johanna
finden einen Platz für die Kennenlernspiele und teilen diesen per
Slack mit. Wenn alle da sind, laufen wir geschlossen zu den gefundenen
Plätze. Falko zählt die Erstis. Laura und Nadine ordern mögichst
früh entsprechend viel Pizza für ca. 13:30 Uhr (ggf. später).

Erstis Während des Wartens mögichst begrüßen und \emph{in Gespräche einbinden}.

\subsubsection{Kennenlernen}

Wir erklären, dass es zwei Gruppen gibt und teilen die Erstis enstprechend
auf. Es darf noch beliebig die Gruppe gewechselt werden. Die Gruppen
teilen sich auf die verschiedenen Plätze auf. Die Erstis schreiben
sich Namensschilder. Wir spielen wir die üblichen Kennenlernspiele:
\begin{itemize}
    \item In Ecken stellen
    \item Sortiert aufstellen
    \item Bingo (finde Menschen, mit bestimmter Eigenschaft)
    \item Nach Bedarf:
          \begin{itemize}
              \item Atomspiel
              \item gordischer Knoten
          \end{itemize}
\end{itemize}
Johanna und Johannes leiten die Spiele.

Kurz bevor die Pizza kommt suchen wir uns einen Platz zum Pizzaessen.
Es folgen erste Infos:
\begin{itemize}
    \item Programm, Kontakt
    \item Fahrräder mitbringen!
    \item Mittwoch Morgen Teller, Tasse und Besteck mitbringen.
    \item Cocktailabend Freitag
    \item Es gibt Erstishirts (Verkauf bei Grillen durch Fachschaft)
    \item Kleidung für Olympia
    \item Bild hochladen für KIT-Card
\end{itemize}
Es folgt das Pizzaessen. Luca, Michael, Johannes und Wendy nehmen
nebenbei die Daten (Name, Email, Cocktailabend, Pizza, Einverständnis
die Daten in Airtable zu speichern) der Erstis in Airtable auf und
sammeln das Geld ein. Die Zeit bis zur Campus-Tour wird mit Gesprächen
und wenn nötig Activity überbrückt. Preise für die Erstis:

\medskip{}

\begin{tabular}{ll}
    Pizza \& Sonstiges             & 5€ + 4€\tabularnewline
    Cocktailabend mit/ohne Alkohol & 10€ / 7€\tabularnewline
    Erstishirt bei Fachschaft      & 5€\tabularnewline
\end{tabular}


\subsubsection{Campus-Tour}

% TODO (alt)
Zur Campus-Tour werden die Erstis gleichmäßig in Gruppen aufgeteilt.
Jede Gruppe soll von mindestens 2 Tutoren geführt werden. Die Erstis
können ihre Taschen in den Seminarräumen lassen, aber nach der Führung
und vor dem Abendessen wieder abholen.

Der Anfang der Tour ist für jede Gruppe gleich:
\begin{description}
    \item [{15:00}] Einteilung in Gruppen\setlength{\itemsep}{0pt}
\end{description}
Jede Gruppe sollte das folgende Programm, aber \emph{in unterschiedlicher Reihenfolge} absolvieren:
\begin{itemize}
    \item Infobau
        \begin{itemize}
            \item Abgabekästen
            \item Info-Fachschaft
            \item Info Bib mit GruppenRäumen
            \item Atis- Accounts machen
            \item Seminarräume
        \end{itemize}
    \item Hörsaal am Fasanengarten
        \begin{itemize}
            \item HM 1 für Infos und LA 1 für Infos(freitags)
        \end{itemize}
    \item Audimax
        \begin{itemize}
            \item GBI und Programmieren
        \end{itemize}
    \item Mensa
        \begin{itemize}
            \item Karte kodieren
            \item Linien-System vorstellen
            \item Auf Cafeteria hinweisen
            \item Anekdotisch: Aufgrund von Stromausfall gab es im letzten Semester ein paar mal Essen umsonst
        \end{itemize}
    \item Bibliothek
        \begin{itemize}
            \item LernRäume
            \item Führungen in der ersten Semesterwoche (dauer ca. 30 min Montags bis Freitags 10-14 Uhr)
        \end{itemize}
    \item AKK
        \begin{itemize}
            \item Kaffee
            \item Günstiges Bier und Glühwein
            \item Bar schichten machen/mithelfen
            \item Partys/Schlonze
            \item Alternativ: AKK Führung (noch nicht angefragt)
            \item o-Phest findet hier statt
        \end{itemize}
    \item Daimler und Benz HS
        \begin{itemize}
            \item Ana 1 und La 1 Mathe(Daimeler) und La 1 mittwochs für infos(Benz)
            \item es ist egal was man hört nur richtige Klausur schreiben
        \end{itemize}
    \item Hertz
        \begin{itemize}
            \item IAM für die Mathematiker die nicht Programmieren hören
            \item Uni Kino
        \end{itemize}
    \item Studeinbüro
        \begin{itemize}
            \item falls noch was zu klären ist geht man am Dienstag mit ein paat Tutoren
                noch hin
        \end{itemize}
    \item Mathebau
        \begin{itemize}
            \item Abgabekästen
            \item FSM
            \item Seminarräume und Lernecken
            \item Räume für studneten sind nach innen (falls ihr nach außen geht für
                ein Tutorium z.b. seid ihr falsch)
            \item Anekdotisch im 3. OG sind die Germanisten
        \end{itemize}
    \item SCC
        \begin{itemize}
            \item Hinweis: Führung am Freitag um 14:30 Uhr
        \end{itemize}
    \item Gaede Hörsal
        \begin{itemize}
            \item IAM (für Mathe die nicht proggen machen
        \end{itemize}
    \item Gerthsen Bei Bedarf
    \item Neue Chemie
        \begin{itemize}
            \item Ana 1
        \end{itemize}
\end{itemize}

\subsubsection{Abendessen}

% TODO (alt)
Wir nehmen nicht am Grillen der Fachschaft teil. für das Abendessen
ist ab 19:00 im Oxford Pub für Turing von Moritz und im Pfannestiel
für Lovelace von Joshua reserviert. Am Ende sollten wir uns sagen
lassen, wie viel Geld wir insgesamt ausgegeben haben, dies für kommende
Reservierungen abschätzen können.

David geht um 18:00 zum Treffen der Fachschaft (HS -101).



\subsection{Dienstag}

% TODO Tabelle (alt)
\begin{spacing}{0.8}
    \begin{tabular*}{1\columnwidth}{@{\extracolsep{\fill}}>{\raggedright}p{0.1\columnwidth}>{\raggedright}p{0.45\columnwidth}>{\raggedright}p{0.35\columnwidth}}
        10:00 & \textsf{\footnotesize{}Formalitäten / KIT-Karten} & \textsf{\footnotesize{}Studentenhaus}\tabularnewline[0.3em]
        \textsf{\footnotesize{}12:15} & \textsf{\footnotesize{}Mittagessen} & \textsf{\footnotesize{}Vogelbräu (Turing), Oxford Pub (Lovelace)}\tabularnewline[0.3em]
        \textsf{\footnotesize{}14:00} & \textsf{\footnotesize{}FBI} & \textsf{\footnotesize{}Treffpunkt für Nachzügler: AKK 13:30}\tabularnewline[0.3em]
        \textsf{\footnotesize{}16:00} & \textsf{\footnotesize{}Lago \& Park} & \tabularnewline[0.3em]
        \textsf{\footnotesize{}18:00} & \textsf{\footnotesize{}O-Phest} & \textsf{\footnotesize{}AKK}\tabularnewline[0.3em]
    \end{tabular*}
\end{spacing}

\subsubsection{Formalitäten}

% TODO (alt)
Alle ohne KIT-Karte treffen sich um 10:00 am Studentenhaus. Alexander
und Wendy bringen alle, die noch Formalitäten zu klären haben, zum
Studeinbüro. Danach wetterabhängig überbrückencken der Zeit bis Mittagessen
draußen oder im Seminarraum.

\subsubsection{FBI}

Alle Fachbereichsinformationen beginnen um 14 Uhr in diesen Räumen:
\begin{tabular}{ll}
    Bachelor Mathematik & Neue Chemie   \\
    Bachelor Informatik & Audimax       \\
    Master Mathematik   & Infobau, -102 \\
    Master Informatik   & Infobau, -101 \\
\end{tabular}

Für Lehramt gibt es gesondert am Mittwoch Programm.

% TODO (alt)
Während des Mittagessens werden für alle FBIs Menschen gefunden, die
uns benachrichtigen, wenn sie vorbei sind. Nach dem Mittagessen gehen
wir geschlossen zu den FBIs. Treffpunkt für alle, die direkt zur FBI
kommen, ist vor dem Audimax. Diese Erstis werden von Fabian, Louis,
Laura, Wendy, Falko und dem Banner abgeholt. Während der FBI treffen
sich die Tutoren in Seminarraum -1.012. Hörsäle für die FBI:\medskip{}

\subsubsection{Stadtführung}

% TODO - und zwar ein dickes

\subsubsection{O-Phest}

% TODO



\subsection{Mittwoch}

% TODO Tabelle (alt)
\begin{spacing}{0.8}
    \begin{tabular*}{1\columnwidth}{@{\extracolsep{\fill}}>{\raggedright}p{0.1\columnwidth}>{\raggedright}p{0.45\columnwidth}>{\raggedright}p{0.35\columnwidth}}
        \textsf{\footnotesize{}09:30} & \textsf{\footnotesize{}Gemeinsames Frühstück} & \textsf{\footnotesize{}Seminarraum}\tabularnewline[0.3em]
        \textsf{\footnotesize{}11:00} & \textsf{\footnotesize{}O-Rallye} & \textsf{\footnotesize{}Campus, Zentrale im Seminarraum}\tabularnewline[0.3em]
        \textsf{\footnotesize{}11:30} & \textsf{\footnotesize{}Lehramtinformation} & \textsf{\footnotesize{}Engelbert-Arnold-HS 11.10}\tabularnewline[0.3em]
        \textsf{\footnotesize{}20:00} & \textsf{\footnotesize{}Kneipentour} & \textsf{\footnotesize{}TP Marktplatz (Turing), Europaplatz (Lovelace)}\tabularnewline[0.3em]
    \end{tabular*}
\end{spacing}

\subsubsection{Gemeinsames Frühstück}

% TODO (alt)
Wir treffen uns mit allen interessierten Erstis ab 9:30 Uhr zum Frühstücken
im Seminarraum.

160 Brötchen werden von Laura organisiert. Jonas, Falko, Johannes
und Joshua bringen Waffeleisen mit. Moritz, Svenja, Luca und Joshua
bringen Waffelteig mit. Belag bringen die Erstis mit. Peter und Philip
bringen einen Wasserkocher und Titia und Johanna eine French-Press-Kanne
mit.

\subsubsection{O-Rallye}

% TODO (alt)
Es wird angekündigt, dass Donnerstag Abend ein Spieleabend stattfinden
wird und Erstis Spiele mitbringen können.

Die Fragebögen für die O-Rallye werden von Moritz in der Fachschaft
abgeholt und dann an die Erstis (zur Selbstorganisation) übergeben.
Es sollten immer Tutoren für Rückfragen o.ä. im Seminarraum sein.
Auf Atis-Accounts hinweisen. Die Lehrämtler werden auf ihre Information
hingewiesen (entsprechende Aufgaben übernehmen). Auf Kneipentour am
Abend hinweisen.

\subsubsection{Lehramtinformation}

% TODO (alt)
Alle Lehramt-Studenten werden von Moritz zur Information gebracht
und ggf. auch wieder abgeholt.

\subsubsection{Kneipentour}

% Tabelle https://docs.google.com/spreadsheets/d/1Ea5M858ijKzbtYuySIMmbNUEUUtv-4jGpAmmq59vIvc/edit#gid=0
% Hierfür soll eventuell auch noch ein getrenntes Übersichts-Dokument entstehen

% TODO (alt)
für die Kneipentour treffen sich Turing am Marktplatz und Lovelace
am Europaplatz. Wir bilden Gruppen und verteilen uns auf die Lokale.
Geeignete Kneipen sind:
\begin{itemize}
    \item Zero (heiße Schokolade, Cocktails)
    \item Scruffys (irisches Bier)
    \item Marktlücke (Prosecco :/ etc)
    \item Brauhaus (hausgemachtes Bier)
    \item Carlos Cocktailsbar (Cocktails, duh!)
    \item Zapfkönig (Bier zum Selbstzapfen)
    \item Kap (Wein, Cocktails)
    \item Phono (Craftbier, Cocktails)
\end{itemize}
Danach wird nach Belieben in andere Lokale gewechselt (Stövchen?)
und die Gruppen können zusammengeführt werden. Das wechseln und der
weitere Verlauf wird engmaschig per Chat koordiniert. Wir versuchen
im Shotz zu enden.



\subsection{Donnerstag}

% TODO Tabelle (alt)
\begin{spacing}{0.8}
    \begin{tabular*}{1\columnwidth}{@{\extracolsep{\fill}}>{\raggedright}p{0.1\columnwidth}>{\raggedright}p{0.45\columnwidth}>{\raggedright}p{0.35\columnwidth}}
        \textsf{\footnotesize{}11:00} & \textsf{\footnotesize{}Brunch} & \textsf{\footnotesize{}Café Emaille (Turing), Café Bleu (Lovelace)}\tabularnewline[0.3em]
        \textsf{\footnotesize{}13:30} & \textsf{\footnotesize{}O-Lympia} & \textsf{\footnotesize{}Forum}\tabularnewline[0.3em]
        \textsf{\footnotesize{}20:00} & \textsf{\footnotesize{}Spieleabend} & \textsf{\footnotesize{}Seminarraum}\tabularnewline[0.3em]
    \end{tabular*}
\end{spacing}

\subsubsection{Brunch}

% TODO

\subsubsection{O-Lympia}

% TODO (alt)
Um 13:30 meldet Moritz die Anzahl unserer Erstis bei der O-Lympia-Orga
an. Alle anderen kommen um kurz vor 13:30 zum Forum. Als Ausweichprogramm
bieten wir Klettern im DAV, Kino, Schwimmen und Spiele im Z10 an.

\subsubsection{Spieleabend}

% TODO (alt)
Wir treffen uns um 20:00 Uhr in den Seminarräumen zum Spieleabend.
Fabian und Pablo besorgen Knabberzeug und ggf. Getränke. Tutoren bringen
Spiele mit (siehe Airtable). Eventuell Ausklingen des Abends in einer
Bar.



\subsection{Freitag}

% TODO Tabelle (alt)
\begin{spacing}{0.8}
    \textsf{\footnotesize{}}%
    \begin{tabular*}{1\columnwidth}{@{\extracolsep{\fill}}>{\raggedright}p{0.1\columnwidth}>{\raggedright}p{0.45\columnwidth}>{\raggedright}p{0.35\columnwidth}}
        \textsf{\footnotesize{}11:30} & \textsf{\footnotesize{}Abschlussveranstaltung} & \textsf{\footnotesize{}HSaF}\tabularnewline[0.3em]
        \textsf{\footnotesize{}12:30} & \textsf{\footnotesize{}Mittagessen} & \textsf{\footnotesize{}Mensa}\tabularnewline[0.3em]
        \textsf{\footnotesize{}14:30} & \textsf{\footnotesize{}SCC-Führung} & \textsf{\footnotesize{}SCC}\tabularnewline[0.3em]
        \textsf{\footnotesize{}20:00} & \textsf{\footnotesize{}Cocktailabend} & \textsf{\footnotesize{}K1-Bar}\tabularnewline[0.3em]
    \end{tabular*}{\footnotesize\par}
\end{spacing}

\subsubsection{Abschlussveranstaltung}

% TODO

\subsubsection{Mensa}

% TODO (alt)
Wir gehen gemeinsam mit unseren Erstis in der Mensa essen. Einige
müssen vermutlich noch ihre Karte kodieren und aufladen.

Nach dem Mittagessen gehen wir in den Schlossgarten.

\subsubsection{Aktivitäten}

% Beim SCC ist immer noch keine Führung gebucht
% TODO

\subsubsection{Cocktailabend}

% TODO (alt)
Teilnahme nur nach Anmeldung bis Donnerstag. Detaillierte Planung
erfolgt über entsprechenden Slack-Kanal. Die für den Einkauf zuständigen
Tutoren gehen Freitagvormittag in die Metro. Die Vorbereitung beginnt
ab 17 Uhr, wir können ab ca. 15:30 in die Bar. Tutoren der ersten
Schicht sind ab 19:00 da. Die letzte Schicht bleibt bis zum Schluss
und Räumen im Anschluss alles auf, das in der Nacht aufgeräumt werden
muss. Schichtplan wird auf Airtable geführt. Titia und Peter bringen
Erstis vom Infobau zur Bar. Auf dem Hadiko-Gelände gibt es Hinweiszettel.



\subsection{Samstag}

% TODO Tabelle (alt)
\begin{spacing}{0.8}
    \textsf{\footnotesize{}}%
    \begin{tabular*}{1\columnwidth}{@{\extracolsep{\fill}}>{\raggedright}p{0.1\columnwidth}>{\raggedright}p{0.45\columnwidth}>{\raggedright}p{0.35\columnwidth}}
        \textsf{\footnotesize{}10:00} & \textsf{\footnotesize{}Mädelsbrunch} & \textsf{\footnotesize{}Mathebau}\tabularnewline[0.3em]
    \end{tabular*}{\footnotesize\par}
\end{spacing}

\subsubsection{Mädels-Brunch}

% TODO

\subsubsection{Mathe-Treff}

% TODO

% TODO Wandern?
% TODO O-Philm?

\end{document}
